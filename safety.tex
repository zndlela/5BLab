% !TEX root = 5Blman.tex
% 2012.12.03 first use in git vcs
% 2009.1.10 inserted list environment to replace sections    
 
\chapter{Laboratory Safety}
Much of the ``fun'' and adventure of learning physics is in doing interesting and relevant demonstrations and experiments in the laboratory. To make this possible we must work together to create a SAFE laboratory environment. Laboratory safety is a primary concern of the Department and your instructor. Precautions to ensure safety have been taken in setting up this lab to the extent possible. However, potential hazards still exist. The best general advice for avoiding these is to use your common sense to work safely, be mindful of others' safety, and do not venture into unknown territory if potential hazards are involved or suspected. Specifically, you should be aware of the following hazards and take care to avoid them.

%\section*{Mechanical Hazards}
\begin{enumerate}	
{\Large \item \textbf{Mechanical Hazards}}

Demonstrations and experiments in mechanics invariably use massive objects which are either dropping, oscillating, or rotating at a high speed.  For example, masses oscillating on a spring seem to have a high likelihood of falling off and landing on someone's foot. When using masses, start by experimenting with small mass values and with small displacements, or when possible, set things up so that the masses are close to the floor. Also, take care to secure masses before rotating them. Wear eye protection when experiments involve eyelevel projectiles.

%\section*{Thermal Hazards}
{\Large \item \textbf{Thermal Hazards}}

The main concerns here are open flames, like a Bunsen burner or candles, hot plates, and the use of boiling water. Some things to keep in mind are: (i) be sure to keep your clothing and hair away from open flames, (ii) always assume that hot plates are still hot when you go to pick them up, (iii) boiling water is easily spilled, and when contained, can result in an explosion spraying hot water and steam across the room. Steam exiting from a narrow passage such as the top of a flask should also be handled carefully. Wear eye protection when appropriate.

%\section*{Electrical Hazards}
{\Large \item \textbf{Electrical Hazards}}

Electrical shock hazards are largely determined by the amount of current that flows through the body as a result of coming in contact with a voltage source. Some things to keep in mind are: (i) have all circuits checked by your instructor before connecting the current source, (ii) remember, 90 volts on a dry cell can be as dangerous as 5,000 volts on a discharge tube, (iii) high currents can cause hand burns and start fires, (iv) if instrumenttation is not working, have the instructor check it out. You should not check fuses or attempt to repair the instrument, (v) do not operate power supplies and other sources beyond the recommended levels.

%\section*{Lasers and Light Hazards}
{\Large \item \textbf{Lasers and Light Hazards}}

The lasers you will use in this lab ($< 5mW$ He-Ne or diodes) are about as dangerous as bright sunlight; it is good practice to keep your beam confined to your lab table. Never point the beam toward a person's head. The main concern here is possible damage to the retina of the eye. Another potential light source hazard is associated with UV emitting sources, but at this level of study, such sources are generally not used. Most of the gas discharge tubes, that you will use are contained in a Pyrex glass tube which absorbs most of the UV radiation.

%\section*{Ionizing Radiation}
{\Large \item \textbf{Ionizing Radiation}}

Radiation ($\alpha, \beta, \gamma$, x-ray, neutron) which is capable of ionizing matter is a serious safety concern. Sources that are used by students at this level of lab work are generally very low in activity ($ <1 pCi$) and are generally exempt from rigorous monitoring. In spite of dealing with only low level sources in this course, we will still follow what are considered good procedures for handling radiation sources. In this spirit, all sources of ionizing radiation should be used with the following safety measures in mind:

\begin{itemize}
	\item Time -- minimize the time of exposure.
	\item Distance -- take advantage of the inverse-square law.
	\item Shielding -- contain radiation at the source by using appropriate shielding to absorb it.
	\item Do not eat or drink in any lab that is used or has been used to work with radioactive sources.
	\item Handle all radioactive sources with tongs.
\end{itemize}
 
 
The Department maintains protocols, including safety measures that must be taken for all lab procedures that you will follow when working with radiation sources. These protocols are available for student review. Also be aware that any device that requires a 10,000 volt source or higher is a potential source of x-rays. A good example of such a source is a discharge tube that is powered with spark or induction coil, a common physics demonstration.

%\section*{Hazardous Substances}
{\Large \item \textbf{Hazardous Substances}}

The use of hazardous substances in this lab are kept to a minimum. The Department has a hazard communication program, and Material Safety Data Sheets are available from your lab instructor or physics stockroom for any hazardous substance that is used in this lab. Some substances to be aware of are : (i) mercury from a broken thermometer, (ii) asbestos, and (iii) carbon dioxide (dry ice) -- remember $CO_2$ is more dense than air and settles out and may be a problem when stored in a small room.

\end{enumerate}
%\newpage
%\includegraphics*[width=\textwidth,trim=120 80 80 120,clip]{5bgraf/pslabgrid} 
%\cleardoublepage

\endinput
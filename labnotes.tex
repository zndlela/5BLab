%--------------------------------------------------------------------------
% !TEX root = 5Blman.tex
% labnotes.tex
%--------------------------------------------------------------------------
% 2013.01.08 separated out labrpt and performance levels
% 2012 Dec
% 2009 Jan. Instructor labnotes

\chapter{Instructor Labnotes}
%\begin{comment}
These notes from 5B were originally compiled to fill a void for instructors teaching the lab for the first time, or for instructors teaching it after a long hiatus when equipment, procedures, labs \dots may have changed. More importantly it is more a series of reminders and observations of some things to consider for the lab along with some suggestions how to get students through the lab. Over the years, a few labs have changed dramatically, others hardly at all. There are lots of places where the manual may lack clarity or sufficient guided assistance for students. Any suggestions you have about \emph{any aspect} of the manual will very likely be an improvement. So, let me (zn) know how your experiences can be incorporated into these notes or the manual itself.
%\end{comment}
%--------------------------------------------------------------------------
\section*{Introductory Lab}
The first lab is not assigned any particular activities. You may want to use it to check your registration roll and give students an idea of what to expect.

I usually stress that the lab is an important part of the course, but that it is not very tightly tied to the lectures nor to their textbook. Different instructors may be on different topics and the textbook may be ordered differently from the lab.

I find it a useful time to go over such things as course requirements, the lab schedule, grading, the lab report \ldots

Several labs refer to ``error and uncertainty analysis''. This would be a good time to let students know what you have in mind and define for them what you expect for error and uncertainty analysis. I try to keep it simple and give examples of percent difference and percent uncertainty. You will probably have to repeat your explanations several times during the semester, but at least students will have heard it a least once.

See Appendix \ref{a:rptformat} for a suggestion of a simple lab write up form. The sample focuses on four major parts of a report, the Introduction--Abstract, Preparation, Data Analysis--Observation, and Comment-Conclusion. It's really simple, but student always ask how should the lab be written up. I've tried many variations, including lab notebooks, but this simple form at least provides an answer to the question, ``What about lab reports''?
%--------------------------------------------------------------------------
\section{Lab 1 -- Electrostatics}
Results for this lab can literally vary with the weather. It might be overkill, but if students wash their hands before handling the equipment, this may decrease the amount of skin oils transferred to the cloths and rods. You may want to provide a brief explanation of the electroscope and warn students against overcharging them. Some of the gold leaves are damaged, so advise students to use the proof plane and show them how to use it when they are transferring charge directly to the electroscope.

A Van deGraft Generator and Wimshurst machine are placed in the lab if you want to use them. Both of them can be quite impressive as demos. I haven't done the hair raising demo in some time, but you will need an insulating platform to do it.

Students this early in the course may find electrostatic polarization and electrostatic induction confusing, even after your best explanations. I've found it useful to use step-by-step diagrams to highlight these phenomena as proceses and require students to do the same in their reports.

%--------------------------------------------------------------------------
\section{Lab 2 -- Electric Fields and Potentials}
For many students, this lab will be the first time they have ever used an electronic meter. To make things go more smoothly, I go into some detail to show them how to use the digital multimeter and explain what each button indicates. You'll probably have to repeat much of this again in later labs, but it will be good reinforcement. I've found that going through a dry run, then actually measuring, say the, voltage output of the VAC is useful. 

Let students know it's easier to outline the ``plates'' on the transparent plastic sheet before putting it in the tray. Then, students can tape the plastic grid sheet to the bottom of the tray, cover it with water, and ``squeegee'' out the bubbles. Alternatively, students can lay the plates onto the water covered sheet, then ``dimple'' an outline of the plates with the probe. It may help students to use the probes by guiding them through sample potential measurements, say at the equipotential line = 1 volt.

I have students do at least two configurations, and three if depending on time --- the parallel plate configuration, the dipole configuration (using the two round cylinders), and a choice using the parallel plates and a cylinder. As an additional learning aid, you may want students to display their samples on the projector and give a brief explanation of what they did, what they found, and how they interpret their results.

One last thing, I ask students to calculate the electric field $E = \Delta V/\Delta d$ for three locations between the parallel plates and one location outside the plates. It helps to do sample calculations on the board, as students may still have difficulty distinguishing between electric potentials, electric voltage, and electric fields.

%--------------------------------------------------------------------------
\section{Lab 3 -- Electrical Energy}
This lab uses an analog volt and current meter. You will probably have to explain how to use the meters, especially how to read the scales correctly. I work in the idea of ``parallax'' and emphasize how important it is to face the meter directly, rather than try to read it from an angle.

This might also be a good time to briefly introduce the idea of circuit diagrams by simply redrawing the block diagram as a schematic. One idea I stress for the next labs that use a voltmeter is this: \textsl{hook the voltmeter up last}. It's the easiest thing to hook up, but also the easiest thing to get wrong.

You may have to remind students to convert energy in cal to Joules (so that $E_{Thermal} \sim E_{Electrical}$), check and recheck the wiring before turning on the power, and keep the current constant at 2.2 A. For the last point, students often read the wrong scale (I refer to them as the 1 and 3 scale), so if their results are totally off, check how they read their current meter.

For the current semester (2010) the emphasis has changed slightly to focus on the mechanical equivalent of heat. You may still prefer to just compare the heat loss/gain in which students must convert the energy to \textsl{cal} or \textsl {J} for the comparison.

%--------------------------------------------------------------------------
\section{Lab 4 DC Circuits Part I}
Students will likely appreciate a brief discussion of the color code and how to read a resistor to obtain its resistance value. I suggest to students that they set up a table which has resistor values determined form (1) the resistor color code, (2) direct measurement of the resistor using an ohmmeter, and (3) calculation of the resistance using Ohm's Law.

You will probably have to explain again how to use the digital multimeter and the analog meter if you have not done it previously. This time students actually use the resistance measurement mode. Consider a sequenced exercise (or steps) that students perform directed by you in which they measure a battery voltage. Then they connect the resistor to a battery, a switch, and ammeter to measure the current. Finally, following the precept,\textsl{ hook up the voltmeter last}, use the voltmeter to measure the voltage across the resistor and across the battery. Hopefully, after this, students will feel more comfortable with using the equipment and doing so correctly.

As part of a pre-lab assignment to introduce students to the next lab (Lab 5: DC Circuits II) you might ask students to perform one or more of these tasks for the values given in the manual:
\begin{itemize}
	\item set-up a current equation at a junction
	\item set up two loop equations for voltage (the two ``inside'' loops and or the ``outside loop'') 
	\item solve the three equations for the unknown currents
\end{itemize}

%--------------------------------------------------------------------------
\section{Lab 5 DC Circuits Part II}
Students may find this lab difficult, especially if they are uncomfortable with multiloop circuits and the instruments. By this time, they might not have been introduced to Kirchoff's rules yet or not had adequate time to use them in problems. A brief discussion illustrating Kirchoff's current and voltage rules using the lab circuit can help, including generation of the current and voltage equations.

Be aware that many students will try to ``compare'' the measured and calculated currents by simply placing the measured values into the circuit equations, then ``solve'' the equations, and note the both sides of the equations are unequal. There are lots of variations on this. 

When measuring the internal resistance \emph{r}, students can get different results from using 
$r  =  \frac{E - V_T}{I} = \frac{E - IR}{I}$. The problem is that the resistance found from measuring the voltage across and the current through the resistor (Ohm's law method), is likely not the same as using the color code resistor value, or the ohm meter value of the resistor. It may be simpler just to use measurements of the emf, the terminal voltage, and the current to obtain the internal resistance.

%--------------------------------------------------------------------------
\section{Lab 6 Electromagnetic Induction}

%--------------------------------------------------------------------------
\section{Lab 7 Electromagnetic Radiation}
Although this is an exploratory lab, you still may want to have students turn in a lab write-up consisting of their observations, sketches, and comments. You should conduct the microwave demo for your lab section, guaranteeing that it will work at least once. The klystron tubes are old, get really hot, and their performance might fail at any time. As an exercise, I give students a set of numbers taken from a measurements of the maxima found when the hardboard is slid along the length between the two horns. From the results, I ask students to determine the wavelength of the microwave radiation and its frequency. [See Table: \ref{t:microwavepeaks} on page \pageref{t:microwavepeaks}].

\begin{table}[htbp]
\begin{minipage}{0.4\textwidth} 
\centering
\caption{Microwave Peak Signal} \label{t:microwavepeaks}
\begin{tabular}{l l l}\toprule
\# & X(cm) & Signal\\
\midrule
1.	&	2.84 &	max\\
2.	&	3.59 &	min\\
3.	&	4.31	 &	max	\\
4.	&	5.09 &	min\\
5.	&	5.86 &	max\\
6.	&	6.57	 &	min\\
7.	&	7.36 &	max\\
8.	&	8.14 &	min\\
9.	&	8.85 &	max\\
10.	& 	9.55 &	min\\
\bottomrule
\end{tabular}
\end{minipage} \hfill
\begin{minipage}{0.5\textwidth} 
	If you decide to use the values in the table, generate your own, or skip this entirely, be sure to explain to students how the wavelength and frequency can be determined.
\end{minipage}
\end{table}

%--------------------------------------------------------------------------
\section{Lab 8 Reflection and Refraction}

%--------------------------------------------------------------------------
\section{Lab 9 Thin Lenses}
I usually give a brief introduction to the lens equation, converging, and diverging lenses, and ray tracing.

You may want to demo how to use all the pieces in the tray to form an image. This goes quickly, which most students will figure out on their own.

I start out by having students determine the focal length of the converging lens by imaging an object "located at infinity", which for us is $\sim$100 times the focal length of the lens and more. For this, two light bulbs are set up at both ends of the lab. Students can use either bulb to ``image'' the bulb with their lens in a holder and measure the image distance - which is also the focal length in this case. 

Then I have them do it again. This time with the room still darkened, I open the blinds at the back of the room and ask students to image anything they can see outside the window. Many students may still be surprised that the image is upside down, even if you have already described the ray tracing.

However, finding the virtual image for a converging lens is not clear. I set up the ``light'' object inside the focal length of the lens then ask a volunteer to find the image on the side that students ``expect'' to find it. I also explain that this virtual image cannot be found or displayed on the screen where they expect to find  it. The image is on the ``other'' side, the side from which the light originates. I place a piece of paper or the screen on that side, and no one is hardly surprised that the image still does not appear, since the light source is blocked. Then I ask two or three students to peer through the lens looking toward the light source and describe what they see - which should be the enlarged image.

Only then do I launch into the explanation of the parallax method telling them to replace the ``light'' object (illuminated round and pointed arrows) with one of the metal pointers along the way. As they can ``see'' the image through the lens, the other pointer is used to find the location of the image. I suggest to them to set the ``image'' pointer a little higher than the ``object'' pointer. Looking through and around (or over) the lens they can sight the virtual image and the image pointer. Then move the image pointer forward and backward while moving their head slightly to the left and right. At some location they will see the image pointer and the image move in unison with each other rather than opposite directions. That location of the image pointer should be the location of the virtual image.

This can be both difficult to explain and to demonstrate. %So another method is suggested below which does not use parallax.

For the focal length of the diverging lens, have students put both lenses (converging and diverging) in one lens holder. The image will be real for the set of lenses we are using and the image distance will be much longer than students expect. So I encourage students to explore a little and try longer distance when they report they cannot see anything.


A few years ago C. Newcomb fashioned two pointers for use in determining the image of a virtual image for a converging lens. These two pointers are referred to as the ``straight'' (3--5 mm diameter by 5 cm rod) and ``bent pointer'' (3 -- 5 mm diameter by 9 cm rod with three right-angled bends). These should be better than the current pointers.

To find the image of a divergent lens,  I think it's better to separate the convergent and divergent lens by a constant distance $d$ then adjust the screen or lens distances until a real image appears. 

%--------------------------------------------------------------------------
\section{Lab 10 Vision}
%--------------------------------------------------------------------------
\section{Lab 11 Optical Instruments}

This lab uses three (3) converging lens with focal lengths which can be roughly described as short ($\sim$ 5 cm) , medium ($\sim$ 15 cm), and long ($\sim$ 27 cm).  Once students select the correct lenses for the three instruments (simple magnifier, telescope, and microscope) there is usually no problem in viewing the image. The problem is usually how to determine the image size from the physical set up.

I explain to students that they can look ``through'' and ``around'' the objective  lens at the same time. Then they can compare the number of say, lines of a ruler, seen through the lens with the number of lines of the ruler.

%--------------------------------------------------------------------------
\section{Lab 12 Diffraction and Interference}
In the past we have always done geometric optics (labs 8 -- 11) before wave/physical optics. Be prepared to make adjustments in case the order changes.

This lab uses the 1 mw HeNe laser. Students are suppose to measure the wavelength of this beam taking only measurements of the screen to length distance, the separation of the interference fringes, and the dimension of the slits.

Be aware that students often confuse the two pairs of slits as a ``double slit''. You might want to point this out to them and be sure that they distinguish the two slits. Also they should be able to tell, within a pair, what comprises the single slit and what comprises the double slit.

You will likely have to cover the operation and use of the linear microscope. Since this is the greatest source of uncertainty, it's important that students know how to use it. This can take a lot of time to explain and to demonstrate, which we don't have. At the minimum I let students know that the linear scale shows 1 cm divisions, 1 complete rotation on the rotary knob is 0.1 cm, and one unit division (say from 0 to 1 on the rotary knob) is 0.01 cm. Then I have students measure the diameter of a dot that they place on a piece of paper. First though, I have them adjust the scope so that the cross-hairs are in focus, then adjust the scope so that the image of the dot is in focus.

Measuring the slit width can be a problem. I suggest that students start at one edge of a pair of slits, record the location then slowly move to each each of the slit and record the location - so they'll have four locations, say 1, 2, 3, 4. They they can determine the slit separation $d$ by the values 3--1, and 4 -- 2.  The slit width, $D$ is then 2 -- 1, and 4 -- 3. Explain to them the problem of backlash and that they must always move in one direction.

To determine $\Delta y$, I suggest to students that they measure  the distance $y$ between $n$ bright fringes, then $\Delta y = y/n$.

If students have not covered diffraction and interference in their classes yet, a very brief explanation of the equation for the double slit ``equation'' might be useful. But all along I've explained to students that they MUST read the manual in advance, and the relevant chapters in their text (which I point out in advance), especially since the labs and lecture/recitations may not be in synch. 

Finally, I ask students to take particular caution to know where there laser beam is going, so that it should not bounce off the slit holder or something else or enter the space of another group.

%--------------------------------------------------------------------------
\section{Lab 13 The Hydrogen Spectra}
This lab may well occur before atomic or quantum physics is discussed, and you may find it too time consuming to do anything more than a cursory introduction.

I've found it more productive to spend most of the time on how to use the instruments -- the diffraction grating and the spectroscope. Then connect the diffraction grating equation and the Balmer equation to what students will be measuring in lab.

For the diffraction grating I make sure students know how to determine $d$, the width of each slit or line in the grating from the number of lines/mm. I also explain that the grating is a precision instrument and that they should handle it by its edges and not put their finger prints on the glass.

When I get to the spectroscope I give a summary description of the instrument -- collimator, telescope, eye piece, rotating table, main scale -- A and B (in degrees), vernier scale (in arc minutes), and adjusting knobs. Then I go into a bit more detail. I don't know if the order makes any difference, but the explanations/descriptions seem to help. I ask students to  ``follow'' me as I skip from one part to the another. It's helpful if the spectroscope is oriented the same for everyone, so I have them point the collimator away from them (away from their belly) and point the base-table adjustment knobs pointing toward them (into their belly). From this orientation it's easier to describe parts of the instruments and adjustments.

This is what I express to students for adjusting the spectroscope. This should probably be part of the lab instructions.
\begin{enumerate}
	\item Obtain a clear view of the crosshairs.\\
- Look into  the eyepiece and move it in and out until the crosshairs are clear.

	\item Adjust the objective lens.\\
- Point the telescope at a distant object (across the room, say).\\
- Adjust the objective lens until the image of the object is in the plane of the crosshairs. Translation: both the crosshairs and the image should be in focus.\\
- Do not change the adjustment of the objective lens after this calibration

	\item Adjust the collimator\\
- Place a light source in front of the collimator slit.\\
- Look toward the collimator through the adjusted telescope. You should see and image of the illuminated slit.\\
- Adjust the collimator lens until the image of the slit is in the plane of the crosshairs. Translation: both the slit and the crosshairs should be sharp and distinct.\\
- Adjust the slit so that it is vertical and narrow -- not too wide, not too thin.

	\item After the calibration is completed, rotate the telescope only by moving the barrel, not by handling the eyepiece, the objective lens, or the collimator.
\end{enumerate}
You may have to check each setup and make suggestions on the slit width, orientation, crosshairs, sharpness....

Ask students to make angular measurements for at least two (2) orders on both sides of the zeroth order. They should be able to see three (3) orders. You may want to forewarn them that the 3rd order blue-violet may begin before the 2nd order red ends. Sometime during the lab at least one group might have a question about this. I've used it as an opportunity to find it on one set-up and have other groups view it on the set-up.

At some point in your descriptions you may want to encourage students to select either the A or B scale and give an example of the use of the vernier scale. Students will have the most difficulty reading the vernier, so any help you can give them will be well spent. I've used a diagram where the angular reading is set for 11 deg 18 min, then used the same (similar) diagram to read 11 deg 48 min. Students may ask where does the 48 come from when there are only 30 units on the vernier. I've just given the brief ``practical'' answer that when the zero of the vernier passes the half-mark (11.5 deg in this case) add 30 min.

I also suggest to students that they rotate their table so that the main scale (A or B) is set at 0 deg or 180 deg rather than some arbitrary position, although any position works.

Warn students that when the telescope is aligned directly opposite the slit (the A or B vernier is at 0 or 180 degrees), the ``color'' they see is \emph{not} the 1st or 2nd order ``red'' : it is simply an image of the slit.
%--------------------------------------------------------------------------
\section{Lab 14 Nuclear Decay and Half-life}
So that students use the geiger counter properly, I have them go through a mock trial run with me with the power off. They turn the knobs, press the buttons, imagine the counts, but the power remains off. Then they perform a real run, going through the power up, knob adjustments, measure the background radiation for 1 minute, then power down. After that, they have a good idea how to use the counter correctly. You may or may not want to describe how the geiger counter works.

Most likely, you will not have much time to work with the $\alpha, \beta, \gamma$ sources. So, I assign groups to concentrate on one of the sources, then write their results on the blackboard so that everyone can copy them. They can then make their assessments.

For the half life measurements, I ask students to do 32 or 34 successive two minute intervals. I explain how to determine the half life by two methods and require that they do both: (1) by plotting values similar to ones they will generate, and (2) by calculating the half life from two rates separated by some time interval.

\textbf{Note:} I have also tried having students make one (1) measurement every half-hour or two (2) measurements every 15 minutes, then use the exponential equation to determine the time constant.

%--------------------------------------------------------------------------
\begin{comment}

\chapter{Laboratory Report Format} \label{a:rptformat}
\section{Laboratory and Results Overview}
Name (and Lab Partners):\hspace{2.61in} Date:\\
Section:. \hspace{3.75in} Day:\\
The report should also have a title which is usually the laboratory name and number.

\noindent \textbf{Lab Objective:} Write a single sentence or short paragraph, in your own words, stating the objective of the laboratory.\\
\textbf{Lab Overview:} Summarize in a single sentence or short paragraph the outcome of your lab experiment.
\begin{itemize}[itemsep=0pt]
	\item Often this will be a brief statement of how closely your observations or measurements agreed with or were consistent with predictions from a hypothesis, theory, or formula.
	\item Many labs involve observation of phenomena, so that your results might list or state your new understanding of the physical phenomena that you gained by doing the lab.
\end{itemize}

\section{Procedure}
\textbf{Equipment:} Provide a sketch of the setup of the major pieces of equipment and their arrangement.\\
\textbf{Experimental Procedure:} Summarize in a sentence or brief paragraph any exceptional procedures, techniques, or methods used during the lab.

\section{Data \& Analysis}
Report your data using tables, drawings, and graphs. Give an appropriate name or title to your tables, plots, and graphs. Example: Suppose you are measuring voltage across and current through a small bulb and record those values. A   poor title is, ``V vs. I'', or ``Voltage vs. Current''. A better title could be, ``Non-linear Resistance of a 12 V bulb''. Look in your text for examples of table and graph names. Be sure to include units for your table columns and include labels and units for the axes of all plots and graphs.

\section{Interpretation \& Conclusion}
Answer all questions required by the instructor and or posed in your lab manual. Where applicable or appropriate, comment on the accuracy and precision of your measurements, include your uncertainties, and identify possible sources of imprecision and uncertainty.

%--------------------------------------------------------------------------
\chapter{Physics 5A/B: Lab Performance levels}
\section{Grading Rubric}
A few years ago, several instructors worked on revising aspects of the 5A course. One of the outcomes was a suggested rubric for grading some of the assignments in the lab and discussion sections. The rubric is reprinted below and may be of some use for labs and assignments where such grading is appropriate.
\begin{table}[htbp] \caption{Lab Performance Levels} \label{t:performance}
\centering
\begin{tabular}{p{2in}p{2in}p{2in}} \toprule
Score 9 - 10		&Score 8 - 9		&Score 7 - 8\\
\midrule
Student work in lab and in the report demonstrates an in depth understanding of scientific concepts and their applications.
	&Student work in lab and in the report demonstrates an in depth understanding of scientific concepts and their applications.
	&Student work in lab and in the report demonstrates an in depth understanding of scientific concepts and their applications.\\
	\midrule
All work is clear and complete and exceeds expectations. It demonstrates skill in formulating strategies for measurements and observations, and the ability to make logical and appropriate arguments.
	&The work is generally clear and complete and meets expectations. It shows evidence of skill in formulating strategies for measurements and observations, and the ability to make logical and appropriate arguments
	&Most work is clear and complete but does not meet all the expectations. It shows some skill in formulating strategies for measurements and observations, and the ability to make logical and appropriate arguments\\
	\midrule
Discussions make strong connections to events outside the laboratory environment including personal experiences and or topics in other scientific disciplines.
	&Discussions establish adequate connections to events outside the laboratory environment including personal experiences and or topics in other scientific disciplines
	&Discussions establish some connections to events outside the laboratory environment including personal experiences and or topics in other scientific disciplines\\

\bottomrule
	
Score 6  7
&Score 5 - 6\\ \midrule

Student work in lab and in the report demonstrates limited understanding and application of scientific concepts.
	&Student work in lab and in the report demonstrates little or no understanding and application of scientific concepts.\\
	\midrule
Work is incomplete and not presented clearly. It shows limited skill in formulating strategies for measurements and observations, and the ability to make logical and appropriate arguments.
	&Work shows almost no progress toward completion and presents vague and irrelevant information. It shows little or no skill in formulating strategies for measurements and observations, and the ability to make logical and appropriate arguments.\\
	\midrule
Discussions make minimal or no connections to events outside the laboratory environment including personal experiences and or topics in other scientific disciplines
	&Discussions do not establish connections to events outside the laboratory environment including personal experiences and or topics in other scientific disciplines.\\
	\bottomrule
\end{tabular}
\end{table}

\end{comment}
 
%--------------------------------------------------------------------------
\endinput
%--------------------------------------------------------------------------

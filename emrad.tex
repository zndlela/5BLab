%--------------------------------------------------------------------------
% !TEX root = 5Blman.tex
% emrad.tex
% 2013.01.07 changed to 2col format
%--------------------------------------------------------------------------
\chapter{Electromagnetic Radiation}

\begin{multicols}{2}
%---------------------------------------------------------------------
\section{Purpose}
  The purpose of this laboratory exercise is to explore both the wave and photon nature of electromagnetic radiation.  In the process you will manipulate and observe several distinct types of electromagnetic radiation ranging from microwaves to x rays, thereby gaining a hands-on familiarity with these entities.

%---------------------------------------------------------------------
\section{Preparation}
  Read the sections in your text and become familiar with the electromagnetic spectrum before coming to lab.  Note the definitions of the various types of electromagnetic waves.  Pay special attention to the effects of these waves, the manner in which they are created, and how they may be detected.

\paragraph{Short quiz}
  Be prepared to take a short quiz for which you need to be familiar with the major categories of electromagnetic waves and their properties.
%---------------------------------------------------------------------
\section{General Information}

Your instructor will explain the operation of the numerous pieces of apparatus in the room.  There is only one station for each activity so you will circulate about the room to perform this lab.  You may perform the activities in any order.
Your instructor will also summarize the wave and photon characteristics of electromagnetic radiation.

%---------------------------------------------------------------------
\section{Exploring EM radiation}
Perform each of the following activities.  Be certain to keep good records of your observations and to answer the specific questions posed.  Note anything that you observe in addition to what is mentioned in this write up.  Such observations may be at least as important as those you are asked to make.

\subsection{Microwaves}
Microwaves are the highest-frequency electromagnetic waves that can be produced by currents in circuits.  They generally have a frequency that is the same as the oscillation frequency of the circuit.

\begin{enumerate}
	\item Sketch the microwave apparatus, labeling the major components.  Note that the only evidence for the microwaves is given by the meter connected to a detection device.
	\item Can you feel the microwaves?  Why don't they cook you like a microwave oven would?
	\item Explore the microwaves for evidence that the wavelength is on the order of the size of the wave guides that carry the microwaves (about 4 cm).  Describe your method and observations.  Calculate the frequency of the microwaves assuming they have a 4.00 cm wavelength. \par
	Alternatively, use the data in Table \ref{t:microwavesignal} to determine the microwave wavelength and frequency.
	\item Explore the microwaves for evidence of polarization.  Describe your method and observations.  Is polarization a characteristic of waves?  Explain briefly.
\end{enumerate}
	
\subsection{Infrared (IR)}
Infrared radiation is usually produced by thermal motion and the vibration and rotation of atoms and molecules.  Its frequencies overlap with the upper end of the microwave range and extend to the lower end of the visible range hence the name infrared or ``below red.''
\begin{enumerate}
	\item Can you see infrared radiation with your eyes?  How do you detect it in this exercise?  Your skin absorbs about 98\% of the infrared radiation that falls on it.  What "color" is your skin in the infrared?
	\item Manipulate the reflectors to determine whether infrared radiation and visible light follow the same laws of reflection.  Perform a similar experiment with the large lens.  Do your results support the contention that infrared and visible light behave similarly?
\end{enumerate}
	
\subsection{Visible Light} 
Visible light is defined to be the part of the electromagnetic spectrum to which the eye normally responds, producing nerve signals.  Visible light can be produced by a wide variety of processes.
\begin{enumerate}
	\item Visible Laser Light
	The name "LASER" is an acronym for Light Amplification by Stimulated Emission of Radiation.  Lasers emit electromagnetic waves that have a very pure frequency and wavelength and which are coherent (all the waves are in phase).  Our lasers emit narrow beams of light and are not dangerous to your vision, but should be treated with caution (like an unloaded gun) since some lasers can easily blind you if the beam enters the eye directly.
	\begin{enumerate}
		\item Pass the laser light through a narrow slit and describe what you observe.  How are your observations consistent with the wave nature of electromagnetic radiation?
		\item Observe the hologram image.  How can you tell that it is a true three-dimensional image?
		\item List several applications of lasers.  Explain how each application is related to the pure frequency and coherent nature of laser output.
	\end{enumerate}
	
	\item Polarization of Visible Light
	\begin{enumerate}
		\item Devise an experiment that demonstrates the polarization of visible light. Two methods are effective and should be explored. The first is the passage of light through certain materials and the second is the reflection of light.
		\item Describe a specific use of a polarizing material.
	\end{enumerate}
	
	\item Visible Spectra	
		Use the diffraction gratings to observe the spectra of the light sources in the large black box.
	\begin{enumerate}	
		\item Sketch the spectrum of each of the five light sources.
		\item What characteristic (if any) of each spectrum is "quantized?"  By quantized we mean that only certain values are observed.
		\item Why does each gas have a different spectrum?
		\item Briefly explain why the filament's spectrum is not quantized (continuous rather than discrete bands of color).
	\end{enumerate}
\end{enumerate}

\subsection {Photoelectric Effect}	
The photoelectric effect is the only observation in today's laboratory exercise that cannot be explained by the wave nature of electromagnetic radiation alone.  It can be explained by the existence of photons (your instructor should discuss the nature of photons or particles of light).
\begin{enumerate}
	\item Describe the operation of the photoelectric apparatus.
	\item Devise an experiment with red and blue filters that implies the photon energy of blue light is greater than the photon energy of red light.
	\item Pick some feature of the photoelectric effect that cannot be explained by waves alone and elaborate on how the wave picture is insufficient to describe this feature.
\end{enumerate}

	
\subsection {Ultraviolet Radiation (UV)}
\begin{enumerate}
	\item Ultraviolet radiation has higher frequencies and hence higher photon energies than visible light.  Many UV characteristics can be explained by the higher photon energy.  One example is the damage done to biological cells by UV.
	\item Fluorescence
	\begin{enumerate}
		\item Observe fluorescence with the black light and the mineral samples provided.
		\item Explain the process of fluorescence in terms of photon energies and atomic excitations and de-excitations.
	\end{enumerate}
	\item A weak source of UV is one with low intensity (a relatively small number of photons emitted per second).  Even weak sources of UV can be good sterilizers and pose a hazard to skin and other living tissue.  Explain why in terms of photon energy.
\end{enumerate}
	
\subsection {X-rays}
X-rays are produced when energetic electrons strike a material.  Inner shell electrons are ejected from some atoms when the incoming electrons strike them.  When another electron fills the hole left, an x-ray is produced.  Cathode ray tubes (such as those in televisions and computer monitors) have energetic electrons that cause a screen to glow.  X-rays are also produced and must be shielded to protect the observer.  Our cathode ray tube is not shielded so you can observe the x-rays it produces.  The intensity of the x-rays is too small to be harmful.
\begin{enumerate}
	\item Describe how you observe the x rays produced by the cathode ray tube.  Does their detection imply the existence of photons?  Explain.
	\item Where do the x-rays seem to originate?
	\item Demonstrate that the x-rays can be shielded against.
	\item How do the damaging properties of x-rays relate to their photon energy?
	\item What precautions are used to protect humans from x-rays produced by televisions and x-ray machines?  Note that all protection involves various combinations of shielding, distance, and time limitation of exposure.
\end{enumerate}

%---------------------------------------------------------------------
\section{Conclusions}
  For which types of electromagnetic radiation did you observe wave characteristics?  For which did you observe photon characteristics?\\
%\hrule 
%\begin{table}[b]
%\begin{minipage}{0.4\textwidth} 
%\centering
%\caption{Microwave Peak Signal} \label{t:microwavesignal}
%\begin{tabular}{l l l}\toprule
%\# & X(cm) & Signal\\
%\midrule
%1.	&	2.84 &	max\\
%2.	&	3.59 &	min\\
%3.	&	4.31	 &	max	\\
%4.	&	5.09 &	min\\
%5.	&	5.86 &	max\\
%6.	&	6.57	 &	min\\
%7.	&	7.36 &	max\\
%8.	&	8.14 &	min\\
%9.	&	8.85 &	max\\
%10.	& 	9.55 &	min\\
%\bottomrule
%\end{tabular}
%\end{minipage} \hfill

\begin{center}
%\begin{table}[b]
\begin{tabularx}{0.8\linewidth}{@{}XXr@{}}
	\hline
	\# & X(cm) & Signal\\
	\hline
1.	&	2.84 &	max\\
2.	&	3.59 &	min\\
3.	&	4.31 &	max	\\
4.	&	5.09 &	min\\
5.	&	5.86 &	max\\
6.	&	6.57 &	min\\
7.	&	7.36 &	max\\
8.	&	8.14 &	min\\
9.	&	8.85 &	max\\
10.	& 	9.55 &	min\\
\end{tabularx}
\mtcaption{Microwave Peak Signal} \label{t:microwavesignal}
%\end{table}
\end{center}

%\begin{minipage}{0.5\textwidth} 
The values shown in Table \ref{t:microwavesignal} display the results from a microwave setup. The displacement values, x, are measured along a straight line between the microwave transmitter and receiver. The signal values represent the locations where the voltage strength reaches a minima or maxima. For example, the value x = 2.84 cm indicates the location where the voltage reading is a maxima whereas the location x = 3.59 cm represents a location where the voltage reading is a minima. From these values you can determine the wavelength.
%\end{minipage}
%\end{table}

\end{multicols}
%--------------------------------------------------------------------------
\endinput
%--------------------------------------------------------------------------

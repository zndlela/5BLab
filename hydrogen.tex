%--------------------------------------------------------------------------
% !TEX root = 5Blman.tex
% optical.tex
%--------------------------------------------------------------------------
% !TEX root = 5Blab.tex
\chapter{The Hydrogen Spectrum}

\section{Purpose}
  The purpose of these laboratory exercises is to measure the wavelengths of hydrogen spectral lines in the visible part of its spectrum.  This will also provide you with further hands-on experience in observing interference and diffraction effects produced by a diffraction grating.

\section{Preparation}
  We use a diffraction grating in this exercise. Pay particular attention to the conditions that produce constructive interference for light passed through a diffraction grating.
  
\section{General Information}
The following explains how we can obtain wavelengths using the diffraction grating and how we can calculate those wavelengths from atomic theory.
When light is passed through a diffraction grating an interference pattern gives constructive interference at angles  related to the wavelength  of the light and the separation $d$ of the slits.  The equation for this is

\begin{equation} \label{e:nfere} 
	d\cdot \sin\theta = m\cdot \lambda, \quad m = 0, 1, 2, \dots
\end {equation}

In our experiment we will observe the first order spectrum; that is, the spectrum for $m = 1$.  You will find a value for $d$ and measure the angles  for the observable spectral lines from a hydrogen discharge tube.  Experimental values of  can then be calculated using equation (\ref{e:nfere}).
Atomic theory produces the following equation for the wavelengths of light emitted by hydrogen atoms.  Theoretical values of $\lambda$ are calculated using equation (\ref{e:balmer}).

\begin{equation}\label{e:balmer}
	\frac{1}{\lambda}= R\left(\frac{1}{n_f^2}-\frac{1}{n_i^2}\right)
\end{equation}

where $R = 1.097 \times 10^7/m$ is the Rydberg constant.  The symbols $n_f$ and $n_i$ are positive integers that represent electron orbits in the hydrogen atom.  If an electron makes a transition between orbits then $n_i$ represents the initial orbit and $n_f$ the final orbit.  For light (EM radiation, actually) to be omitted $n_i$ must be greater than $n_f$.  Visible radiation is obtained for orbits that end in the second orbit where $n_f = 2$.  The values of $n_i$ can thus be 3, 4, 5, 6\dots .  This is called the Balmer series.

\section{Atomic Spectra}

\subsection{Activity: Spectrometer calibration}
% Calculate $d$, the separation or distance between slits.
\begin{enumerate}
	 \item 	Your grating has a label on it giving the number of lines per centimeter.  From this calculate the value of $d$, the distance between slits.
	 \item 	The primary function of the spectrometer is to measure the angles at which spectral lines occur.  Your instructor will give you directions for proper adjustment of the spectrometer so that you can obtain the most accurate angle measurements possible.
	 \item 	Place the grating in the center of the spectrometer with its lines (slits) vertical.  The slit of the collimator must also be vertical.  The grating must be perpendicular to the light coming from the collimator.
	\item Adjust the collimator slit size
	
	Look directly at the hydrogen discharge tube through the slit of the collimator.  Move the source if it is not directly in the middle of the slit.  Adjust the size of the slit until it is narrow enough to produce narrow spectral lines, but not so narrow as to reduce the intensity of light to the point where it is difficult to see with the eye.
	\item Align the hydrogen discharge tube with the collimator
	
	Place the hydrogen discharge tube in front of the slit of the collimator.  CAUTION:  Do not touch the high voltage connections or the discharge tube when the power is on.  The high voltage can shock you and the tube becomes thermally hot and can produce burns.
	
	\item Record the zero angle reading of the spectrometer
	
	When looking directly at the hydrogen discharge tube, the spectrometer is at 0 degrees according to equation (\ref{e:nfere}) above.  Your instructor will help you to adjust the spectrometer so that it reads 0 degrees under this condition.
\end{enumerate}

\subsection{Activity: Measuring spectral lines}
\begin{enumerate}
	\item Examine the scales and vernier for measuring angles
	
	The most difficult part of this lab is to properly measure angles.  Carefully examine the angular scales and the vernier scales on the spectroscope.  Note that you will measure minutes of arc (there are 60 minutes per degree) rather than decimal fractions.  Ask your instructor for guidance if you are uncertain how to read the scales.
	\item Measure angles for the first order hydrogen spectral lines
	
	Move the telescope to one side and observe the first order spectrum.  You should be able to see three lines easily and perhaps a fourth with some difficulty.  The colors of the lines are red, green, blue-green, and violet.  Note that the dimmer lines may not appear to have the proper color since color vision fails for dim light, so the blue-green may appear violet and the violet may look gray.  Make the cross hairs in the telescope fall in the middle of each line and record the angle observed.
	\item Repeat the measurement of angles on the opposite side
	% Measure and record the angles as above, but on the other side of zero.
\end{enumerate}
	
\subsection{Activity: Calculating spectral lines}
\begin{enumerate}
	\item Calculate the experimental values for the wavelengths
	
	Average the two angles obtained for each line and then use equation (\ref{e:nfere}) to calculate the wavelength.  Estimate the experimental uncertainties in $d$ and find the total percent uncertainty in each.  Add these percent uncertainties to obtain the total percent uncertainty in your measurements.
	\item Calculate the theoretical values of the wavelengths
	
	Using equation (\ref{e:balmer}) calculate the theoretical values of the wavelengths of the first four lines in the Balmer series.  These have $n_f$ = 2 and $n_i$ = 3, 4, 5, and 6.
	\item Compare experiment and theory
	
	Find the percent difference between the experimental and theoretical values for each of the wavelengths observed. Determine and discuss whether there is agreement within experimental uncertainty. 
\end{enumerate}

\section{Conclusions}
Did today's laboratory exercise succeed in giving you a feeling for interference produced by diffraction gratings and their use as a spectroscopic tool?  What other conclusions can you reach regarding this laboratory exercise?

% \clearpage
%\newpage
%\includegraphics*[width=\textwidth,trim=120 80 80 120,clip]{5bgraf/pslabgrid} 
 
%--------------------------------------------------------------------------
\endinput
%--------------------------------------------------------------------------

%--------------------------------------------------------------------------
% !TEX root = 5Blman.tex
% 5blabmemfront.tex
%--------------------------------------------------------------------------
%  2008.7.30 text for \frontmatter section
%  2009.1.2 modified titlepage, titleGM section

\pagestyle{empty}		% prevent page numbering of title pages
%----half-title page-------------------------------------------------------
\begin{comment}
\vspace*{7pc} %{\fill}
\begin{adjustwidth}{1in}{1in}
\begin{flushleft}
	\HUGE\sffamily Physics 5B \\
	\LARGE\sffamily Laboratory Manual \\
	\vspace{1pc}
	\HUGE\sffamily Electricity, Magnetism, Optics
\end{flushleft}
\end{adjustwidth}
\vspace*{\fill}
%\cleardoublepage
\newpage
\end{comment}

%----titleGM page----------------------------------------------------------
%\begin{comment} % vertical line along lhs
%\newcommand*{\titleGM}{\begingroup% Gentle Madness 
% picture can be inserted directly below Department of Physics... by inserting the \\ cmd then \includegraphics[scale=0.5]{5bgraf/frizzhair} followed by the \par cmd. Should also remove the \vspace... cmd to allow for more room.

\vspace*{\baselineskip} 
%\vfill 

\hbox{\hspace*{0.2\textwidth}% 
  \rule{1pt}{\textheight} 
  \hspace*{0.05\textwidth}% 
  \parbox[b]{\textwidth}{ 
	\vbox{% 
		{\noindent\HUGE\bfseries Physics 5B\\[0.5\baselineskip] 
		Electricity, Magnetism,\\ Optics}\\[2\baselineskip] 
		{\huge\itshape Laboratory Manual}\\[4\baselineskip] 
		{\LARGE Department of Physics and Astronomy}\par
		\includegraphics[scale=0.6]{5bgraf/frizzhair} \par
%		\vspace{0.4\textheight} 
		{\noindent \Large {California State University, Sacramento\\
%		Zolili Ndlela}}\quad(Editor, Spring 2009 revision)\\
		(Spring 2010 revision)}}\\
		[3\baselineskip] 
	}% 		end of vbox 
  }% 		end of parbox  
}% 		    end of hbox 

%\vfill 
%\null 
%\endgroup} 
%\end{comment}

%----title page------------------------------------------------------------
%\begin{comment}
\newpage
\vspace*	{2in}
\begin{center}
	\HUGE\textsf{Physics 5B}\par
\end{center}
\begin{center}
	\LARGE\textsf{Laboratory Manual}\par
\end{center}
\begin{center}
	\HUGE\textsf{Electricity, Magnetism, Optics}\par
\end{center}
\begin{center}
	\Huge\textsf{Spring 2010}\par
\end{center}
\vfill
\begin{center}
	\LARGE\textsf{Zolili Ndlela\\Editor, Spring 2010 Revision}\par
\LARGE\textsf{California State University, Sacramento}\par
\end{center}
 
\vspace*{\fill}
\def\ZUN{Z\kern-0.2em U\kern-0.4em N}%		OK for CMR
\def\ZUN{Z\kern-0.15em U\kern-0.3em N}%		OK for Palatino
\clearpage
%\end{comment}

%----copyright page-------------------------------------------------------
\begingroup
\footnotesize
\setlength{\parindent}{0pt}
\setlength{\parskip}{\baselineskip}
%%\ttfamily
%\textcopyright{} 2008,2009,2010 Zolili U. Ndlela \\
\textcopyright{} 2008,2009,2010 CSUS \\
All rights reserved

CSUS, Sacramento, CA.

Printed in Sacramento 

The paper used in this example may meet the minimum requirements
of the American National Standard for Information 
Sciences --- Permanence of Paper for Printed Library Materials, 
ANSI Z39.48--1984.

\begin{center}
12 11 10 09  \hspace{2em}8 7 6 5 4 3 2       
\end{center}
\begin{center}
\begin{tabular}{ll}
Gene Barnes:						& August 1993 \\
Duane Ashton:					& January 1997 \\
Peter Urone:						& January 1998 \\
Charles Newcomb:              	& January 2003 \\
Reprint PPU edition:				& August 2007 \\
LaTeX editions:					& August 2008, 2009\\
											& January 2010 \\

\end{tabular}
\end{center}
\vfill
*Special thanks is extended to Dr. William DeGraffenreid for proofing the 2008 draft document and to Heidi Yamazaki for all the little things.
\hspace*{\fill}Zolili Ndlela\quad(Editor, 2008-10 revision)
\endgroup

%----quote page-------------------------------------------------------------
\clearpage

\begin{quote}
\textbf{physics,} \textit{n.} [L. \textit{physica} natural science --- Gr. \textit{physika} from neuter plural of \textit{physikos} of nature, from \textit{physis} growth, nature, from \textit{phyein} to bring forth]
  \hspace{1ex} \textbf{1.} a science that deals with matter and energy and their interactions 
  \hspace{1ex} \textbf{2.} [\textit{a}] the physical processes and phenomena of a particular system [\textit{b}] the physical properties and composition of something
%  \hspace{1ex} \textbf{3.} [\textit{pl.}] a report or record of 
%      important events based on the writer's personal observation, 
%      special knowledge, etc.
%  \hspace{1ex} \textbf{4.} a report or record of a scholarly 
%      investigation, scientific study, etc.
%  \hspace{1ex} \textbf{5.} [\textit{pl.}] the record of the proceedings
%      of a learned society 
      \\[0.5\baselineskip]
  \hspace*{\fill} \textit{Webster's New World Dictionary, Online Edition}.
\end{quote}

\vspace{2\baselineskip}

\begin{quote}
\textbf{how radio works.\ }You see, wire telegraph is a kind of a very, very long cat. You pull his tail in New York and his head is meowing in Los Angeles. Do you understand this? And radio operates exactly the same way: you send signals here, they receive them there. The only difference is that there is no cat. \\[0.5\baselineskip]
\hspace*{\fill} Albert Einstein, \emph{when asked to describe radio}
\end{quote}

\vspace*{\fill}

\cleardoublepage

%---TOC---------------------------------------------------------------------
\pagestyle{plain} % resume page numbering

\tableofcontents
\setlength{\unitlength}{1pt}
\clearpage
\listoffigures
\clearpage
\listoftables
\clearpage
%\listofegresults

%----Preface----------------------------------------------------------------

%\chapter{Preface}
%Comments?
%{\raggedleft{\scshape ZN} \\ Sacramento, CA \\ June 2008\par}

%----Introduction-----------------------------------------------------------
%\chapter{Introduction to 5B Lab}
%Comments?

%--------------------------------------------------------------------------
\endinput
%--------------------------------------------------------------------------

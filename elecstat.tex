% !TEX root = 5Blman.tex

%--------------------------------------------------------------------------
\chapter{Electrostatics}
%\pagenumbering{arabic} 		% Start text with arabic 1. Place in 1st chap
%--------------------------------------------------------------------------

\section{Purpose}
The purpose of this laboratory exercise is to give you hands-on experience with static electricity and to allow you to explore some of its characteristics.  The activities you perform will help you to understand a number of concepts that are also explored in lecture, your text, and in homework questions and problems.
%--------------------------------------------------------------------------
\section{Preparation}
 Note the definitions of important concepts in these pages (basic definitions are in bold face in the text).  Pay special attention to the description of the electroscope and its operation.
\paragraph{Short quiz/Prelab}
  Be prepared to take a short quiz at the beginning of lab or to hand in a prelab assignment.  Questions such as, ``How many types of charge are there?" and ``Is charge conserved?" are typical of what may be asked.  A prelab assignment might require you to, ``Explain briefly the process of electrostatic induction", or  ``Prepare an introduction/abstract to the lab". The purpose of the quiz or the prelab is to motivate you to look at and become familiar with pertinent material before coming to the lab.
%--------------------------------------------------------------------------
\section{General Information}
%\begin{framed}
\paragraph{Electroscope and its use} 
Your instructor will describe the various parts of the electroscope and demonstrate its use for you. To avoid damaging the foil just watch carefully as you charge the electroscope and stop if the leaf appears to be in danger (i.e. it accelerates too quickly or reaches angles greater than 50 degrees).  Alternatively, your instructor may show you how to use a proof plane to charge the electroscope.

\begin{quote} \hrule 
\textsf{Caution:} The gold leaf is fragile and will tear away if too great a charge is applied to the electroscope.
\vspace{7pt}
\hrule 
\end{quote}
%\end{framed}
\paragraph{Isolating positive and negative charge.}  You may assume that excess positive charge is left on a glass rod rubbed with silk and excess negative charge is left on a rubber rod rubbed with wool.  (These are not things that you can prove in our laboratory, but they have been established by years of careful experiments.)  Without these assumptions you would not be able to determine whether you are working with positive or negative charge.

\paragraph{Which charges move?}  When separating charges by rubbing, it is usually the electrons (negative charge) that move.  Electrons also move in metals, but in other substances positive charges (ions and molecules) may or may not move---in some situations both negative and positive charges may move. So, a negatively charged object has an excess of electrons, while a positively charged object has a deficiency of electrons.

%--------------------------------------------------------------------------
\section{Activities}
Include ample sketches to help describe your activities and observations. Use $+$ and $-$ signs on your pictures to show where there are excess positive or negative charges.

\subsection{Charging objects}
\begin{enumerate}
	 \item Charge the electroscope by direct contact with a negative charge.  Describe the movement and position of charges during the process.
	 \item Charge the electroscope by direct contact with a positive charge.  Describe the movement and position of charges during the process.
	 \item  Devise, document, and show your instructor an experiment to demonstrate that like charges repel and unlike charges attract.  You may wish to use rods suspended in cradles or rods and the electroscope.
\end{enumerate}

\subsection{Van de Graff generator}
\begin{enumerate}[resume]
	 \item Determine whether the charge on the van de Graaff is positive or negative using a charged electroscope.  
	 \item Describe and explain the motion of charges in the electroscope and how this supports your conclusion.
\end{enumerate}

\subsection{Grounding}
\begin{enumerate}[resume]
 \item To demonstrate the process called grounding, touch a positively charged electroscope to a sink pipe.  Describe and explain what happens.
\end{enumerate}

\subsection{Polarization}
\begin{enumerate}[resume]
	\item Bring a lighted match near positively and negatively charged electroscopes. Describe and explain what happens.
	\item Bring a charged rod near some small bits of paper.  Describe and explain what happens.
	\item Bring a highly charged positive rod near a thin stream of water.  Repeat using a highly charged negative rod.  Describe and explain what happens. Is there anything about water molecules that may enhance the effects you observe?
\end{enumerate}

\subsection{Electrostatic induction}
\begin{enumerate}[resume]
	 \item Your instructor will demonstrate how to charge an electroscope by induction.  Be sure that you can duplicate this yourself. 
	 \item Write your observations and explanations of the motions of charges during each stage of the process.
\end{enumerate}
 
%--------------------------------------------------------------------------
\section{Questions}

Discuss briefly whether or not your observations today give you any information on the following aspects of Coulomb's law.  Describe which activity or activities that provided the information. One way to proceed is to plot displacement of the leaves with distance.
\begin{enumerate}
	\item Coulomb force acts at a distance.
	\item Coulomb force is inversely proportional to distance.
	\item Coulomb force is inversely proportional to distance squared.
	\item Coulomb force is proportional to the product of the charges involved.
	\item Coulomb force acts along a line joining the charges involved.
\end{enumerate}

%--------------------------------------------------------------------------
\newpage
\section*{Notes}
\subsection*{Chart of electron ``donors'' and ``acceptors''}
This table is for your reference. When objects made of different materials are rubbed together, electrons move from materials higher on the list to those lower on the list.

\begin{table}[htdp]
\caption{Chart of Triboelectricity}
\begin{tabular}{cl} \hline
\centering
less tightly bound electrons	
	  		& polyester plastic fabric \\
$|$		  	& rabbit fur \\
$|$			& human hair \\
$|$			& glass \\
$|$			& polyacrylic clear plastic (``Lucite'') \\
$|$			& wool \\
$|$			& quartz \\
$|$			& cat fur \\
$|$			& lead \\
$|$			& silk \\
$|$			& human skin \\
$|$			& cotton \\
$|$			& wood \\
$|$			& amber \\
$|$			& rubber (balloon) \\
$|$			& polystyrene (plastic foam packing material) \\
$|$			& polyvinylchloride (PVC pipe) \\
$\downarrow$ 	& polyethylene (grocery bag) \\
more tightly bound electrons
	& polyperfluorethylene (``Teflon'') \\
\hline

\end{tabular}
\end{table}
%--------------------------------------------------------------------------
\endinput

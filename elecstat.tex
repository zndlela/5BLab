% !TEX root = 5Blman.tex

%--------------------------------------------------------------------------
\chapter{Electrostatics}
%\pagenumbering{arabic} 		% Start text with arabic 1. Place in 1st chap
%--------------------------------------------------------------------------
\begin{multicols}{2}
\section{Purpose}
The purpose of this laboratory exercise is to provide hands-on experience with electrostatic phenomena and to allow you to explore some of its properties.  The activities you perform will help you to understand a number of concepts that are also explored in lecture, your text, and in homework questions and problems.
\paragraph{Short quiz/Prelab}
  In this and future labs, your instructor may have you take a short quiz at the beginning of lab or hand in a prelab assignment.  Such assignments  will generally be connected to the current laboratory and are designed to help prepare you to handle the relevant material prior to coming to the lab.
%--------------------------------------------------------------------------
\section{Preparation}
% Note the definitions of important concepts in these pages (basic definitions are in bold face in the text).  Pay special attention to the description of the electroscope and its operation.
Read the chapter and sections in your text regarding electric charges. You should be well aware that there are only two types of charge, which are denoted as positive ($+$) and negative ($-$). Elementary charges are protons which are tightly held to the atomic nucleus and electron which exist in \emph{loose orbits} surrounding the nucleus -- so it is only the negative electron charge which is free to move and can actually be transferred. Both positvely and negatively charged ions and molecules are also mobile and both can be transferred. 

A useful way to think of a positively charge object is that it has a deficiency of negative charge, and that a negatively charged object has an excess of negative charge. Although only electrons actually move (if you ignore charged ions and molecules), it is still a useful model to regard all positive charge as equally mobile. Direct contact is the means that most objects are charged by conduction.

Charges are bound to insulator type materials and \emph{stick} to them, but can be ``rubbed off''. Charges flow easily in conductor type materials, like metals.

%--------------------------------------------------------------------------
\section{General Information}
%\begin{framed}
\paragraph{Electroscope and its use} 
This lab uses a gold-leaf electroscope, a device that detects the presence and magnitude of charge. 
%Your instructor will describe the various parts of the electroscope and demonstrate its use for you. 
To avoid damaging the foil the best method to transfer charge to the electroscope is to use the \emph{proof plane} -- rub the charged rod onto the proof plane, then touch the proof plane to the electroscope. The leaf should not accelerates too quickly or reach angles greater than 50 degrees.

\begin{quote} \hrule 
\textsf{Caution:} The gold leaf is fragile and will tear away if too great a charge is applied to the electroscope.
\vspace{7pt}
\hrule 
\end{quote}
%\end{framed}
\paragraph{Isolating positive and negative charge.}  You may assume that excess positive charge is left on a glass rod rubbed with silk and excess negative charge is left on a hard rubber rod rubbed with wool.  (These are not things that you can prove in our laboratory, but they have been established by years of careful experiments.)  Without these assumptions you would not be able to determine whether you are working with positive or negative charge.

\paragraph{Which charges move?}  When separating charges by rubbing, it is the electrons (negative charge) that move or transfer.  Electrons also move in metals, but in other substances positive charges (ions and molecules) may or may not move---in many situations both negative and positive charges may move. 
%So, a negatively charged object has an excess of electrons, while a positively charged object has a deficiency of electrons.
\end{multicols}
\hrule
%--------------------------------------------------------------------------
\begin{multicols}{2}
\section{Activities}
Include \texttt{ample} sketches to help describe your activities and observations. Use $+$ and $-$ signs on your pictures to show where there are excess positive or negative charges. 

To improve charging by rubbing, avoid skin contact with the ends of the rods where charge transfer is to occur, wash your hands, clean the rods with an alcohol wipe, or neutralize them by sticking in one of the ``neutralizing'' jars placed in the lab.

\subsection{Charging objects}
With these tasks, plot displacement of the leaves with distance to give you some idea of the strength of electric charge and how this may correspond to coloumb's law.
\begin{enumerate}
	 \item Charge the electroscope by direct contact with a negative charge.  Describe the movement and position of charges during the process.
	 \item Charge the electroscope by direct contact with a positive charge.  Describe the movement and position of charges during the process.
	 \item  Devise and document an experiment to demonstrate that like charges repel and unlike charges attract.  You may wish to use rods suspended in cradles or rods and the electroscope.
\end{enumerate}

\subsection{Van de Graff generator}
\begin{enumerate}[resume]
	 \item Determine whether the charge on the van de Graaff is positive or negative using a charged electroscope.  
	 \item Describe and explain the motion of charges in the electroscope and how this supports your conclusion.
\end{enumerate}

\subsection{Grounding}
\begin{enumerate}[resume]
 \item To demonstrate the process called grounding, touch a positively charged electroscope to a sink pipe.  Describe and explain what happens.
\end{enumerate}

\subsection{Polarization}
\begin{enumerate}[resume]
	\item Bring a lighted match near positively and negatively charged electroscopes. Describe and explain what happens.
	\item Bring a charged rod near some small bits of paper.  Describe and explain what happens.
	\item Bring a highly charged positive rod near a thin stream of water.  Repeat using a highly charged negative rod.  Describe and explain what happens. Is there anything about water molecules that may enhance the effects you observe?
\end{enumerate}

\subsection{Electrostatic induction}
\begin{enumerate}[resume]
	 \item Your instructor will demonstrate how to charge an electroscope by induction.  Be sure that you can duplicate this yourself. 
	 \item Write your observations and explanations of the motions of charges during each stage of the process.
\end{enumerate}
\end{multicols}
%--------------------------------------------------------------------------
\section{Questions}

%Discuss briefly whether or not your observations today give you any information on the following aspects of Coulomb's law.  Describe which activity or activities that provided the information. 
%One way to proceed is to plot displacement of the leaves with distance.
Consider your observations and activities of today's lab. Discuss them with your lab partners. Your instructor may ask you to discuss briefly one or more of the statements below.
\begin{enumerate}
	\item Coulomb force acts at a distance.
	\item Coulomb force is inversely proportional to distance.
	\item Coulomb force is inversely proportional to distance squared.
	\item Coulomb force is proportional to the product of the charges involved.
	\item Coulomb force acts along a line joining the charges involved.
	\item Make a general statement describing the behavior of a neutral electroscope when a charged object is brought near to, but not touching, it.
	\item Summarize how you can tell by using a test rod whether an electroscope is positively or negatively charged.
	\item A negatively charged rod is brought nearby a charged electroscope and the leaves of the 
electroscope return to their vertical position.  What can you conclude about the electroscope? 
	\item Describe the major difference between the methods of conduction and induction regarding the actual method of charging the electroscope.
	\item Estimate the typical amount of charge you transfer to the electroscope by conduction.
\end{enumerate}

%--------------------------------------------------------------------------
\newpage
\section*{Notes}
\subsection*{Chart of electron ``donors'' and ``acceptors''}
This table is for your reference. When objects made of different materials are rubbed together, electrons move from materials higher on the list to those lower on the list.

\begin{table}[htdp]
\centering
\caption{Chart of Triboelectricity}
\begin{tabular}{cl} \hline
\centering
less tightly bound electrons	
	  		& polyester plastic fabric \\
$|$		  	& rabbit fur \\
$|$			& human hair \\
$|$			& glass \\
$|$			& polyacrylic clear plastic (``Lucite'') \\
$|$			& wool \\
$|$			& quartz \\
$|$			& cat fur \\
$|$			& lead \\
$|$			& silk \\
$|$			& human skin \\
$|$			& cotton \\
$|$			& wood \\
$|$			& amber \\
$|$			& rubber (balloon) \\
$|$			& polystyrene (plastic foam packing material) \\
$|$			& polyvinylchloride (PVC pipe) \\
$\downarrow$ 	& polyethylene (grocery bag) \\
more tightly bound electrons
	& polyperfluorethylene (``Teflon'') \\
\hline

\end{tabular}
\end{table}
%--------------------------------------------------------------------------
\endinput

%--------------------------------------------------------------------------
% !TEX root = 5Blman.tex
% nuclear.tex
%--------------------------------------------------------------------------
\chapter {Nuclear Decay and Half-Life}

\begin{multicols}{2}
%---------------------------------------------------------------------
\section {Purpose} The purpose of these laboratory exercises is to observe the penetrating ability of three types of nuclear radiation and to measure the half-life of a radioactive substance.

%---------------------------------------------------------------------
\section {Preparation} Your lab instructor will introduce pertinent concepts at the start of the laboratory period.

%---------------------------------------------------------------------
\section {General Information}
Do not eat or drink during this laboratory. Wash your hands after you leave the laboratory. Although the radioactive sources are sealed, as an extra precaution we handle them only with the tweezers provided. Your instructor will explain the hazards of nuclear radiation and the proper procedures for radiation protection.

%---------------------------------------------------------------------
\section {Nuclear decay}

\subsection{Activity: Geiger counter}
\begin{enumerate}
	 \item Turn on the Geiger counter

Adjust the counter to the proper voltage as directed by your lab instructor. Do not exceed the recommended value

	\item Determine background counting rate
	
Do this by allowing the counter to run for two minutes and recording the number of counts. Do this at least three times and average the number of counts.

	\item Determine the ability of  $\alpha, \beta$, and $\gamma $ rays to penetrate materials

Place various absorbers between the source and Geiger counter until the radiation measured in a two-minute interval is reduced to background levels. Do this for each of the three sources provided.

	\item Explain the ability of radiation to penetrate materials

At the discretion of your lab instructor, you may be asked to explain why the different types of radiation have different abilities to penetrate materials.

\end{enumerate}


\subsection{Activity: Measuring the half-life of $^{115}In$}
% Measure the half-life of a radioactive source
Your instructor will explain how the source is created and discuss the meaning of half-life.
\begin{enumerate}
	\item Place the radioactive source in the apparatus as you did the $\alpha, \beta$, and $\gamma$ sources earlier in the lab.

	\item Count for at least 32 successive two-minute intervals, subtracting the average background contribution from each run.

	\item Determine the half-life by finding the time required for the counting rate to decline to one half some earlier value. Since the actual half-life is 54 minutes, you should be able to do this for three or four of your early two-minute counts. Average your values and compare the result with the actual value.
\end{enumerate}
\end{multicols}

\section {Questions and Conclusions}
Were you able to confirm the relative ability of $\alpha, \beta$, and $\gamma$ radiation to penetrate materials?

Did you see evidence that the half-life does not depend on when you start counting? How many half lives must pass before all of a radioactive source is gone? 

 %--------------------------------------------------------------------------
\endinput
%--------------------------------------------------------------------------
